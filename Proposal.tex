\documentclass{article}

% if you need to pass options to natbib, use, e.g.:
%     \PassOptionsToPackage{numbers, compress}{natbib}
% before loading neurips_2021

% ready for submission
\usepackage[final]{neurips_2021}
% to compile a preprint version, e.g., for submission to arXiv, add add the
% [preprint] option:
%     \usepackage[preprint]{neurips_2021}

% to compile a camera-ready version, add the [final] option, e.g.:
%     \usepackage[final]{neurips_2021}

% to avoid loading the natbib package, add option nonatbib:
%    \usepackage[nonatbib]{neurips_2021}

\usepackage[utf8]{inputenc} % allow utf-8 input
\usepackage[T1]{fontenc}    % use 8-bit T1 fonts
\usepackage{hyperref}       % hyperlinks
\usepackage{url}            % simple URL typesetting
\usepackage{booktabs}       % professional-quality tables
\usepackage{amsfonts}       % blackboard math symbols
\usepackage{nicefrac}       % compact symbols for 1/2, etc.
\usepackage{microtype}      % microtypography
\usepackage{xcolor}         % colors

\title{Project Proposal 8420}

% The \author macro works with any number of authors. There are two commands
% used to separate the names and addresses of multiple authors: \And and \AND.
%
% Using \And between authors leaves it to LaTeX to determine where to break the
% lines. Using \AND forces a line break at that point. So, if LaTeX puts 3 of 4
% authors names on the first line, and the last on the second line, try using
% \AND instead of \And before the third author name.

\author{%
	Will Sherrer \\
	Clemson University\\
	wsherre@g.clemson.edu \\
	\And Ashton Sobeck \\
	Clemson University\\
	asobeck@g.clemson.edu
}





\begin{document}
\maketitle
%\section{Submission of papers to NeurIPS 2021}
\section{Proposal} 

Our group wants to explore the effectiveness of Convolutional Neural Networks (CNN) 
  when  images that have been reduced by Principal Component Analysis (PCA) have 
  been input as training.  We aim to use the Canadian Institute for Advanced 
  Research (CIFAR) dataset to conduct PCA on and compare how different amount of 
  features in input images effect the accuracy of a CNN.

\section{Motivation}

Throughout our class of CPSC 8420, we have learned many things. 
One of the most crucial concepts of the class is certainly Singluar Value 
Decomposition (SVD). Using SVD, one can utilize PCA to find the important 
features of data. CNNs are quite effective in identifying patterns within 
images through the use of filters and non-linearity within their structure. 
However, when input images get larger and larger, the network can grow in 
size and in return, require much more computational power to train in a timely manner.

\section{Method}

 The machine learning techniques we plan to implement are PCA and SVD. These will be used to reduce the dimension of an images which the CNN models will be trained and tested on. A technique we could be improving on is seeing if AI models can maintain performance whilst minimizing input data. We believe this could be a good preliminary study on performance vs training sample data. 

\section{Intended Experiments}

Our experiment we are thinking about is designing 2 CNN models with different architectures. This will just test if one architecture is better than another with testing. We then plan to take the CIFAR images and train the models with the original image as a control. We will record the training loss and accuracy as well as testing loss and accuracy. Then we will use PCA to reduce the data further to 20\%, 40\%, 60\%, and 80\% of the original and train both models with the same hyper-parameters for consistency. We will record the training and testing loss and accuracy and plot the data and compare to see if the models are able to learn a reduced image. 

\section{Conclusion}

We found an interesting paper, "Image retrieval method based on CNN and
dimension reduction" (https://arxiv.org/pdf/1901.03924.pdf) that also conducts a study on this topic. They found that while CNN's can still have satisfactory performance, further practical and theoretical studies should be conducted to prove the viability and best parameters. 

\end{document}